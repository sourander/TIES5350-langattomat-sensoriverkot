\documentclass[finnish,utf8,nonumbib,palatino]{KokkolaRaportti}

\usepackage[bookmarksopen,bookmarksnumbered]{hyperref}

% ------------ Omat alkaa ------------
\usepackage{fontspec}
\setmainfont{TeX Gyre Pagella}

% For minted (pygment code syntax highlighting)
\usepackage[outputdir=aux]{minted}

% For long tables that span multiple pages
\usepackage{longtable}

% Minted background color
\definecolor{LightGray}{gray}{0.95}

% Potentially set some defaults, like
\setminted{
  % These two are inluded in my tex-code-minted-block snippet. tex
  % frame=lines,
  % linenos,               % Show line numbers
  bgcolor=LightGray,
  breaklines,            % Wrap long lines
  autogobble,            % Automatically dedent leadings whitespace
  fontsize=\footnotesize % Font size to footnotesize. Alternative e.g. \small.
}

% Colors for TODO-list
\usepackage{xcolor}
\newcommand{\todo}{\textcolor{orange}{TODO }}
\newcommand{\done}{\textcolor{green}{DONE }}
\newcommand{\miss}{\textcolor{red}{MISS }}

% ------------ Omat loppuu ------------

\title{Langattomien sensoriverkkojen soveltaminen ympäristön monitorointiin: Tulivuorenpurkauksen ennustaminen}
\setauthor{Jani}{Sourander}

\begin{document}

\mainmatter

% --------------------------------------------------  %
%                                                     %
%                  Johdanto                           %
%                                                     %
% --------------------------------------------------- %
\chapter{Johdanto}


Tämä raportti on osa Langattomat sensoriverkot -kurssia. Tehtänantona on valita yksi Kandris et al. (2020) esittämistä luokituksista. Kuusi sovellusalueesta tässä työssä keskitytään erityisesti ympäristö (engl. Environmental) -alueeseen, jossa sensoriverkkoja hyödynnetään maaperän, veden tai ilman seurannassa.

Raportissa tarkastellaan kahta tieteellistä artikkelia kurssin luentomateriaalien ja -videoiden rajaamien aiheiden kautta.. Ensimmäinen, \emph{An Internet of Things (IoT) Application on Volcano Monitoring} (Awadallah et al., 2019) \cite{volcano}, esittelee konkreettisen ympäristön monitorointisovelluksen, jossa LoRa-verkkoteknologiaa ja virtapihejä sensorisolmuja hyödynnetään tulivuoren maaperän lämpötilan seurannassa. Raportti tutustuu artikkeli kautta sen arkkitehtuuriin, laitteistoon, viestintäprotokolliin, energiatehokkuuden ja turvallisuuteen. Sovellus edustaa ympäristöluokan (Environmental) sensoriverkkoja ja auttaa ennustamaan tulivuorenpurkauksia.

Toinen artikkeli, \emph{WSN Protocols and Security Challenges for Environmental Monitoring Applications: A Survey} (Adu-Manu et al., 2022) \cite{security}, toimii tausta-aineistona ja tarjoaa laajan katsauksen ympäristön monitorointiin liittyviin protokolliin ja tietoturvahaasteisiin. Tämä kirjallisuuskatsaus täydentää sovellusartikkelin tarkastelua tuomalla esiin turvallisuuteen liittyviä näkökulmia.

\subsection*{Generatiivisen tekoälyn käyttö}

Tekoälyä on käytetty tiedonhaun apuna. Suoritin sekä perinteistä Google Scholar että Gemini Deep Research tiedonhakua. Jälkimmäisessä käytetty kehote ja taulukkomuotoon parsimani tulokset löytyvät liitteistä (ks. Liite \ref{app:deep-research} ja Liite \ref{app:deep-research-results}). Tekoälyä on käytetty myös LaTeX-muotoilun apuna taulukoissa.


% --------------------------------------------------  %
%                                                     %
%                 Volcano                             %
%                                                     %
% --------------------------------------------------- %
\chapter{Tulivuorenpurkauksen ennustaminen}


% --------------------------------------------------  %
%                Artikkelin valinta                   %
% --------------------------------------------------- %
\section{Artikkeleiden haku ja valinta}

Raporttia kirjoittaessa oli tärkeää löytää artikkeli (tai useampi), jossa kuvataan sovellusta eri arkkitehtuurin kerrosten näkökulmista. Tätä valintaa vaikeuttaa se, että kurssin aihe, langattomat sensoriverkot (\emph{engl. Wireless Sensor Networks, WSN}) on hyvin lähellä risteäviä aiheita kuten esineiden internetiä (\emph{engl. Internet of Things, IoT}), joten sopivaa työtä ei välttämättä tunnista otsikosta tai tiivistelmästä. Toinen vaikeuttava tekijä on, että kurssin moduulin katsantakulma on laaja, kattaen vaiheet mittauskerroksesta applikaatiokerrokseen asti. Tämä asettaa laajuusvaatimuksia löydetylle julkaisulle. Näiden haasteiden johdosta tiedonhaun prosessiksi valikoitui tapa, jossa ladataan kymmeniä PDF-julkaisua koneelle, ja nämä käydään silmäillen läpi, karsien ne pois, jotka eivät läpäise vaatimuksia. Haku suoritettiin sekä Google Deep Research että Google Scholar palveluita käyttäen ja hakuprosessi esitellään raportin liitteissä.

Kuten liitteiden kehotteista on luettavissa, alkuperäinen aiherajaus oli \emph{Flora \& Fauna}. Julkaisuja karsiessa valtaosa tekoälyn ehdottamista julkaisuista karsiutui pois, koska niiden pääpaino oli selkeästi IoT:n tai koneoppimisen puolella. Pisimpään harkinnanalaisena pysyi TreeVibes-sensoriin liittyvät kaksi  julkaisua. Ne tunnetaan numeroilla 9 ja 9B liitteissä (ks.Liite \ref{app:deep-research-results}). Deep Resarchin tuottama ehdotus ei sisältänyt alkuperäistä TreeVibes-julkaisua, mutta julkaisu tunnusnumeroltaan 9 käytti kyseistä sensoria ja viittasi tähän työhön lähteissään. Artikkeleiden tarkempi silmäily valitettavasti todisti, että TreeVibes on pikemminkin IoT- kuin WSN-toteutus, joten julkaisut karsiutuivat pois. Tähän mennessä kaikki ladatut PDF-tiedostot olivat joutuneet hylkäyspinoon, paitsi väärää sovellusaluetta, eli ympäristöä, -- tarkemmin tulivuoria -- käsittelevä työ. Lopulta ei ollut muuta vaihtoehtoa kuin tehdä pivot: sovellusalueeksi valikoitui \emph{Environmental}.

Näin sovellukseksi oli valikoitunut kirjoittajien Awadallah, Moure ja Torres-González julkaisu \emph{An Internet of Things (IoT) Application on Volcano Monitoring}. Julkaisu on Espanjan kansallisen maantieteellisen instituutin (IGN, esp. \emph{Instituto Geográfico Nacional}) Karariansaarten geofysiikan keskukselta (esp. \emph{Centro Geofísico de Canarias}). Myös tässä julkaisussa mainitaan IoT useita kertoja -- alkaen otsikosta. Luennoilla Ismo Hakala käsitteli tätä aihetta: termit Wireless Sensor Networks (WSN), Internet of Things (IoT) sekä Cyber-Physical Systems (CPS) ovat läheisesti toisiinsa liittyviä käsitteitä. Sopivalla Google Scholar -haulla (\texttt{"WSN"\ AND\ "IoT-enabled"}) löytyy vuoden 2020 jälkeen yli $13\,000$ tulosta. Kenties IoT nähdään tällöin IP-protokollaperheen tarjoamana toiminnallisuutena, jonka alemmalla kerroksella voi olla tai voi olla olematta kurssin määritelmän mukaisia WSN-laitteita. Käsitellyssä julkaisu muotoilee tämän näin: \emph{"WSN is often a technology used within an IoT system"} \cite{volcano}. Tässä tulivuorten sensorointia käsittelevässä toteutuksessa IoT-tasoa edustaa Raspberry Pi -gatewayltä alkava Internet-yhteys. Alemman mittaustason laitteet ovat kurssin määritelmän mukaisia WSN-mittalaitteita ja hyödyntävät kommunikaatioon LoRa-tekniikkaa. Näihin teknisiin seikkoihin tutustutaan myöhemmissä luvuissa.

Valitun sovelluksen lisäksi raportin aineistoksi valikoitui toinen julkaisu, joka on kirjallisuuskatsaus (engl. review article), joka käsittelee WSN-protokollien turvallisuutta. Kyseinen Google Scholarilla löytynyt lähde, joka tunnetaan liitteissä tunnusnumerolla 26(ks. Liite \ref{app:google-scholar-results}), on otsikoltaan \emph{WSN Protocols and Security Challenges for Environmental Monitoring Applications: A Survey} \cite{security}. Tämä katsaus on mukana vertailukohtana ja tarjoamassa laajempaa pohjaa turvallisuusaiheen käsittelylle.

% --------------------------------------------------  %
%                Yleiskuvaus                          %
% --------------------------------------------------- %
\section{Sovellusartikkelin yleiskatsaus}

Tulivuorenpurkausten ennustaminen ja riskien hallinta edellyttävät monipuolista signaalien seurantaa. Seurantaa tulee tehdä seisomologian eli maanjäristysopin, geodesian eli maanmittausopin ja geokemian näkökulmista. Raportissa \emph{An Internet of Things (IoT) Application on Volcano Monitoring} Awadallah ja kollegat ovat asentaneet kahdeksan matalavirtaista, LoRa-verkkoon kytkettyä lämpömittaria Teide-tulivuorelle Teneriffalla, Kanariansaarilla. He mittaavat fumarolien eli kaasulähteiden lämpötiloja. Perinteisiä mittaustapoja ovat paikan päällä tapahtuva mittaus -- in situ -- tai kaukokartoitusmenetelmät infrapunakameroita käyttäen. Suorat mittaukset ovat tarkkoja, mutta rajoittuvat tyypillisesti pienelle alueelle, sillä tulivuoret ovat tyypillisesti syrjäisiä ja vaikeakulkuisia ympäristöjä. \cite{volcano}

Näiden haasteiden ratkaisemiseksi langattomat sensoriverkot (WSN) ja esineiden internet (IoT) tarjoavat kustannustehokkaan ja energiapihin vaihtoehdon, joka mahdollistaa ympäristön autonomisen valvonnan ilman jatkuvaa fyysistä huoltoa. Vaikka WSN-teknologiaa on aiemmin sovellettu seismiseen seurantaan, sen hyödyntäminen lämpötilan ja kaasujen monitoroinnissa on ollut vähäistä huolimatta näiden muuttujien tärkeydestä vulkaanisen aktiivisuuden arvioinnissa. Tutkimus vastaa tähän tarpeeseen esittelemällä skaalautuvan WSN/IoT-mittausjärjestelmän, joka kestää vulkaamisten alueiden ankarat olosuhteet ja pärjää minimaalisella virralla, paristovirralla. \cite{volcano}


% --------------------------------------------------  %
%                 Toiminnallisuus                     %
% --------------------------------------------------- %
\section{Toiminnallisuus}

\subsection{Arkkitehtuuri}

Tulivuorisovelluksen verkkotopologia noudattaa hierarkkista mallia, joka koostuu Teneriffan osalta kolmesta pääkomponentista: sensorisolmuista (end devices), toistimesta (repeater) ja yhdyskäytävästä (gateway) \cite{volcano}. Kuten kurssin luennolla todettiin, langattomissa sensoriverkoissa etäisyyden kasvaessa lähetystehon tarve kasvaa, jolloin monihyppytekniikka (multi-hop) on usein välttämätöntä energian säästämiseksi ja kantaman pidentämiseksi. Tulivuorisovelluksessa tämä on toteutettu sijoittamalla toistin kraatterin reunalle välittämään viestejä sensorien ja kauempana sijaitsevan yhdyskäytävän välillä, mikä mahdollistaa kommunikaation myös ilman suoraa näköyhteyttä yhdyskäytävään \cite{volcano}.

Laitteistotasolla sovelluksen arkkitehtuuri ilmentää resurssirajoitteisten laitteiden jakoa eri luokkiin. Luentomateriaalin perustella tyypilliset sensorisolmut ovat usein "Class 1" -laitteita (n. 10 kB RAM), mutta tässä sovelluksessa käytetty PIC16LF1788-mikrokontrolleri 2 kB:n keskusmuistillaan sijoittuu alimpaan "Class 0" -kategoriaan, mikä asettaa tiukat rajat ohjelmistolle ja vaatii äärimmäistä energiatehokkuutta. Tämän vastapainona yhdyskäytävänä toimiva Raspberry Pi edustaa tehokkaampaa laiteluokkaa, joka kykenee ajamaan Linux-käyttöjärjestelmää ja hallinnoimaan TCP/IP-liikennettä. Raspberry Pi on kiinni verkkovirrassa ja jossakin mainitsemattomassa, alueella saatavilla olevassa Internet-yhteydessä. \cite{volcano}

Kokonaisjärjestelmä noudattaa kerrosarkkitehtuuria, joka ulottuu fyysisestä datan keruusta pilvipalveluun asti. Kuten kurssin luennolla todettiin, yhdyskäytävä (gateway) toimii siltana ja protokollamuuntimena sensoriverkon ja laajemman verkon (kuten Internetin) välillä. Tulivuorisovelluksessa yhdyskäytävä vastaanottaa LoRa-radiolla lähetetyt mittaukset, tallentaa ne paikallisesti ja synkronoi tiedot erilliseen tietokeskukseen (Data Analysis Center) visualisointia varten \cite{volcano}. Tämä vastaa luentokalvojen esittämää viisikerroksista "Device-Network-Processing-Application Layer-Business" -jakoa, jossa dataa jalostetaan ylemmillä tasoilla loppukäyttäjälle ymmärrettävään muotoon.

\subsection{Protokollat}

Sovellus vaatii tulivuoren ympäristön takia pitkäkantoisen ja vähävirtaisen viestintäprotokollan (\emph{engl. low-power wide-area networks (LPWAN)}). Kolmesta harkitusta protokollasta -- NB-IoT, Sigfox ja LoRa -- ensimmäinen eli NB-IoT käyttää lisensoituja kaistoja ja karsiutui näin ehdokkaista. LoRa valikoitui Sigfoxin sijasta useista syistä: LoRa-päätelaitteet ovat halvempia kuin Sigfoxin, symmetrinen tiedonsiirtonopeus, LoRa on paremmin konfiguroitavissa, ja LoRa ei vaadi kuukausimaksullista tilausta tai sisällä päivittäisiä tiedonsiirtokattoja. LoRa on patentoitu ja käyttää tiedon lähettämisessä modulaatiotatekniikkaa nimeltään Chirp Spread Spectrum modulation (CSS). Tiedonsiirron nopeus on kymmenien kilobittien luokassa. Sen suorituskykyä ja sähkönkäyttöä voi säädää eri konfiguraatioiden avulla, joista mainitaan: \emph{"transmission power (TP), carrier frequency (CF), bandwidth (BW), coding rate (CR), and spreading factor (SF)"}. Tarkka moduulin malli on Ra-02 ja valmistaja Ai-Thinker. Moduuleita testattiin oikeassa ympäristössään ja signaali oli yli 15 desibeliä kohinan paremmalla puolella käyttäen parasta speading factoria -- eli SF 7:aa. Tarkat RSSI-arvot olivat -100 dB ja -105 dB. Antenneina noodeissa on sekä 2-dBi dipoliantenneja ja 9-dBi suuntaavia jagiantenneja ja ne operoivat 433 MHz taajuudella. \cite{volcano}

Tiedonkeryy hoituu siten, että yhdyskäytävänä toimiva Raspberry Pi noutaa (engl. pull) dataa eli lähettää kyselyn sensorinoodeille. Nämä vastaavat paluupostina kyseiseen kysymykseen, oli se sitten pyyntö RSSI-mittauksista tai lämpötilamittauksista. Kaikki kommunikaatio kulkee toistimen kautta. Yhdyskäytävä materialisoi datan SD-muistikortille ja synkronisoi sen data-analyysikeskukseen. \cite{volcano}

Kommunikaatiota varten tiimi kehitti oman protokollan. Protokolla on yksikanavainen ja vain yksi laite kommunikoi kerrallaan. Näin kollisioita ei synny laisinkaan. Protokollassa kommunikointi alkaa aina yhdyskäytävän kyselyllä, jolla tarkistetaan toistimen tila. Jos toistin vastaa, lähetetään varsinainen viesti tai pyyntö, jonka toistin välittää eteenpäin. Kyseessä on siis koodijakokanavointi (engl. code-division multiple access, CDMA), jossa kukin sensorinoodi tunnistetaan yksilöllisellä SF-koodilla. Toistin voi siis välittää viestin vain sille sensorinoodille, jolle se kuuluu, eikä broadcast-lähetyksenä kaikille. \cite{volcano}

Sensoriverkkoa voi jatkossa laajentaa helposti. Ainut, mitä tarvitsee muuttaa, on yhdyskäytävän konfiguraatiotiedosto. Sitä voi muokata ottamalla yhteyden Raspberry Pi -laitteeseen SSH-protokollaa hyödyntäen Internetin yli.\cite{volcano}

\subsection{Teknologiat}

Yhdyskäytävän antenni on 3-dBi suuntaava antenni ja se on kohdistettu toistimen suuntaan. Yhdyskäytävän piirilevy on IoT-ratkaisuissa tyypillinen Raspberry Pi 3B. Raspberry valittiin useista syistä, kuten mahdollisuudesta suorittaa Linux-käyttöjärjestelmää, laajoista liitäntämahdollisuuksista, sekä avoimen kehittäjäympäristön laajasta tuesta. Tutkijoita kiinnostivat myös lisälaitteet Raspberry Shake ja Boom, jotka ovat seismometrejä. Yhdyskäytävän osalta virrankulutuksella ei ole suurta merkitystä, sillä se sijoitettiin verkkovirran saataville. Raspberry ja Ra-02 LoRa-moduuli keskustelevat SPI-sarjaliikenneväylän kautta. Yhdyskäytävä suorittaa Python-ohjelmaa 10 minuutin sykleissä. Muistikortti kantaa myös muita ohjelmia, kuten RSSI-testauksen käynnistävän skriptin, mutta nämä ovat ihmisen käsin ajamia, SSH-yhteyden yli käskyttäen. Yhdyskäytävä maksaa kaikkine osineen 61 EUR.\cite{volcano}

Sensorisolmu on asennettu 60-senttimetrisen PVC-muotiputken eri päihin. Mittauspäässä on alumiininen RTD-lämpötila-anturi. Pieni PT100-sensori on sijoitettu alumiinisen nupin sisään, joka siirtää lämmön sensorille, mutta suojaa sitä korroosiolta. Toisessa päässä putkea on 2 dBi-dipoliantenni, Ra-02 Lora-moduuli, PIC16LF1788-mikrokontrolleri sekä MAX31685-siru, joka on resistanssi-digitaalisignaali--muunnin. Tämä 60 cm pitkä kokonaisuus on työnnetty 40 senttimetrin syvyyteen maahan pystyyn siten, että 20 cm jää näkyville. Maasta törröttää siis pieni PVC-putkenpala, jonka päässä on pieni tupakan kokoluokkaa oleva dipoliantenni. Sensorisolmu maksaa kaikkine osineen 37 EUR. \cite{volcano}

Sensorisolmujen mikrokontrolleriksi valikoitui jo aiemmin mainittu PIC16LF1788, mutta vertailussa oli mukana myös muita, kuten ATMEGA328, STM32F103C8T6 ja PIC18LF25K40. Valittu mikrokontrolleri on näistä virtapihein. Unessa se kuluttaa noin 3 mikrowattia, kun verrokit kuluttavat järjestyksessä 16, 13 ja 46 mikrowattia. Yhdyskäytävä tilaa sensoreilta mittauksia kerran 10 minuutissa, joten nimenomaan unitilan tehovaativuus on merkittävä seikka mikrokontrollerin valinnassa. Se myös kestää -40 -- 125 asteen Celsius-lämpötiloja. Mikrokontrolleri kommunikoi MAX31685-digitaalimuuntimen kanssa SPI-protokollaa eli sarjaliikennettä käyttäen. Virtalähde on AA-kokoinen litiumtionyylikloridiparisto, joka tuottaa 3,6 voltin jännitettä 2400 mAh kapasiteetilla. Verrokkina oli myös Li-Ion ladattava akku, muuta niiden spesifikaationmukainen käyttölämpötila loppuu 60 Celsius-asteeseen. Täten ei-ladattava paristo valittiin, vaikka sen kapasiteetti on pienempi. Vulkaanisessa maaperässä on syitä odottaa 85-asteen lämpötiloja. Näin sekä mikrokontrolleri että sen virtalähde kestävät tyypilliset olosuhteet tulivuorella. \cite{volcano}

Toistin on hyvinkin samanlainen laite kuin sensorisolmu. Kehitysajan säästämiseksi tiimi käytti tismalleen samaa 60 cm PVC-rakennetta, mutta toisessa päässä putkea ei ole lämpötilamittaria vaan 12 V jännitteen sisäänvienti. Toistin on toistuvasti tietoa vastaanottavassa tilassa, eikä täten hyödy unitilan säästöistä samoissa määrin kuin sensoritsolmut. Tämän johdosta sen virtalähteenä ei ole paristo vaan 12 voltin akku ja aurinkopaneeli. Kokonaisuus on sijoitettu kraaterin reunalle siten, että siitä on suora näköyhteys yhdyskäytävään. Toistin on noin kymmenen euroa edullisempi kuin sensorisolmu, koska PT100 ja MAX31685 puuttuvat. Toistin maksaa 27 EUR. Kokonaisuudessaan kaikki komponentit, mukaan lukien 8 sensorisolmua, maksoivat 385 EUR.\cite{volcano}

Kursilla mainittua jonopalvelua kuten MQTT-brokeria ei ole käytössä artikkelin kuvauksen perusteella. Jos palvelin on alhaalla, Raspberry Pi todennäköisesti bufferoi viestejä ja lähettää ne kerralla, kunhan palvelin on saatavilla. Mittauksista muodostetaan kuvaajia analytiikkapalvelimen päässä. Nämä HTML-raportit päivittyvät 10 minuutin välein ja niistä on luettavissa lämpötilamittauksien lisäksi myös RSSI-tulokset sekä pakettien virhemäärät. \cite{volcano}

% --------------------------------------------------  %
%                      Analyysi                       %
% --------------------------------------------------- %
\section{Analyysi}

Sovellus sai yllättävän testin, kun tavallisesta poikkeavat sääolosuhteet aiheuttivat haasteita. Yksi sensorisolmu mittausalueilta A ja B olivat mykkänä useita päiviä. Lumi peitti sekä toistimen että vastaanottimet. Yhden vastaanottimen pariston jännitteentason mittaus on myös mykkänä johtuen rautatason ongelmista. PVC-putki aiheutti myös ongelmia, jotka eivät olleet ilmenneet laboratorio-olosuhteissa. Oli oletus, että 20 cm näkyvä osuus toimisi jäähdyttävänä elementtinä. Laboratoriossa ei oltu mitattu yli 30 asteen lämpötiloja putken sisällä, mutta kenttäolosuhteissa osoittautui, että lämpötila nousi korkeammalle kuin mihin paristot ovat suunnitellut. Näin yhden solmun paristo tyhjeni ennätysajassa. Tämä koski vain ladattuja paristoja, joita oli aluksi käytetty sekaisin ei-ladattavien paristojen kanssa. \cite{volcano} Kokonaisuutena alkuperäistä ratkaisua ei voi nimittää erityisen toimintavarmaksi siinä muodossa, missä se ensimmäisen kerran kentälle vietiin. Jatkokehityksellä applikaation toimintavarmuutta voi parantaa.

Käsitelty artikkeli on ristiriidassa kirjallisuuskatsauksen esittämien turvallisuusvaatimusten kanssa. Tulivuorten lämpötilamittauksissa keskitytään energiatehokkuuteen ja kustannusten minimointiin. Valitussa PIC16LF1788-mikrokontrollerissa on vain 2 KB muistia \cite{volcano}, joten se lasketaan kurssimateriaalin perusteella Class 0 -noodiksi. Siru vie noin 3 mikrowattia tehoa unitilassa ja 70 milliwattia käytössä. Tämä 32 megahertsin suoritin tuskin olisi kykeneväinen ajamaan raskaita turvallisuusprotokollia, kuten vahvaa salausta. Data lähetetään seuraavassa muodossa toistimille: $\text{VIDXX:abc.de,f.gh}$, jossa ID on noodin ID, abc.de on lämpötilatulos eli esimerkiksi 123.22 astetta, ja f.gh on pariston jännite esimerkiksi 3.25 volttia \cite{volcano}. Tämä mahdollistaa hyökkäyksen, jossa hyökkääjä lähettää LoRa-radiolla toistuvasti mittauskäskyjä sensoreille, mikä kuluttaa sensoreiten paristot merkittävästi normaalia toimintaa nopeammin. Kyselytutkimuksessa tämä luokitellaan verkkotason hyökkäykseksi. Istuvimpana terminä on \emph{Generation of false messages}. Mahdollinen ehdotettu puolustuskeino on salaus tai tunnistautuminen. \cite{security} Mikäli datan perusteella tehdään automaattisia tai puoliautomaattisia varoituksia tulivuoren purkautumisesta, vihamielinen valtio voisi vaivattomasti tekaista paketteja ja saada näyttämään mittaustulokset siltä, että saarelle tulisi nostaa hälytystila.

Sensoreiden keräämä data ei kenties ole luottamuksellista, joten salauksen puute ei aiheuta arkaluonteisen tiedon, yrityssalaisuuksien tai henkilötietojen vuotamista salakuuntelun kautta. Suurin haaste on kuitenkin se, että artikkeli ei laisinkaan käsittele mahdollisuutta, että joku häiritsisi verkkoa, syöttäisi vääriä lämpötilatietoja tai vakoilisi dataa. Jää epäselväksi, kuinka hyvin esimerkiksi yhdyskäytävänä toimiva Raspberry Pi on suojattu. Kuvien perusteella se on ulkotilassa olevassa suojalaukussa. Mikäli sen muistikorttia ei ole salattu, mahdollinen hyökkääjä voi vaivattomasti ottaa talteen Linux-käyttöjärjestelmästä mahdolliset SSH-avaimet tai pilvipalveluiden tokenit. On myös epäselvää, kuinka Raspberryn SSH-palvelin on suojattu: onko se jonkin VPN-verkon takana vai suoraan julkiverkossa kiinni. Ainut mainittu turva-asia on LoRan sisäinen ominaisuus: "The resulting signal has noise-like properties, making it harder to detect or jam." \cite{security}


% --------------------------------------------------  %
%                                                     %
%                  Yhteenveto                         %
%                                                     %
% --------------------------------------------------- %
\chapter{Yhteenveto}

Tässä raportissa tarkasteltiin langattomien sensoriverkkojen (WSN) soveltamista ympäristön monitorointiin, erityisesti tulivuorten maaperän lämpötilan seurannassa. Pääasiallisena tutkimuskohteena oli Awadallahin ja kollegoiden (2019) kehittämä sovellus Teide-tulivuorella, Tereriffalla, jossa hyödynnettiin LoRa-verkkoa ja energiatehokkaita mikrokontrollereita. Järjestelmän arkkitehtuuri perustui hierarkkiseen malliin, jossa sensorisolmut, toistimet ja yhdyskäytävä muodostivat itsenäisen verkon haastaviin maasto-olosuhteisiin. Toteutus osoitti, että edullisilla komponenteilla ja optimoidulla virransäästöllä on mahdollista rakentaa toimiva pitkän kantaman mittausjärjestelmä ilman raskasta infrastruktuuria, joskin sääolosuhteiden kestävyydessä on hiomisen varaa.

Analyysissä sovelluksen ratkaisuja verrattiin yleisiin WSN-tietoturvahaasteisiin. Vaikka järjestelmä menestyi energiatehokkuudessa ja kustannussäästöissä, sen turvallisuusmekanismit osoittautuivat puutteellisiksi. Salauksen ja autentikaation puute altistaa verkon viestien väärentämiselle ja palvelunestohyökkäyksille, mikä on merkittävä riski tulivuoren kaltaisen kriittisen kohteen valvonnassa. Raportin johtopäätöksenä on, että ympäristösovelluksissa on löydettävä herkkä tasapaino äärimmäisen virransäästön ja riittävän tietoturvan välillä, jotta järjestelmä on sekä pitkäikäinen että luotettava.

\appendix

\chapter{Deep Research kehote}
\label{app:deep-research}

Alla on kehote, jota käytin löytääkseni mielenkiintoisia WSN-sovelluskohteita valitussa kontekstissa. Tekoälynä Google Gemini Pro, Deep Reseach -tila.

\vspace{1.0\baselineskip} % ----

\noindent
\textbf{Role:} You are an expert academic researcher specializing in Wireless Sensor Networks (WSN) and the Internet of Things (IoT).

\vspace{1.0\baselineskip} % ----

\noindent
\textbf{Task:} Identify 10--20 unique, rare, or surprising ``Flora \& Fauna'' WSN application cases. I am looking for ``out-of-the-box'' topics that go beyond common greenhouse monitoring or livestock tracking.

\vspace{1.0\baselineskip} % ----

\noindent
\textbf{Inspiration for Uniqueness:}

\noindent
\begin{itemize}
    \item \textbf{Fauna:} Monitoring the synchronized flashing of fireflies, tracking highly endangered or elusive species (e.g., pangolins, snow leopards, or deep-sea vents), or bio-logging with miniature sensors on insects/birds.
    \item \textbf{Flora:} Monitoring ``plant intelligence'' or communication (e.g., fungal networks/mycelium), monitoring rare orchid habitats, or tracking the ``heartbeat'' (sap flow) of ancient trees.
    \item \textbf{Extreme (Flora or Fauna) Environments:} Underwater sensor networks for coral reef health or sensors in high-altitude volcanic ecosystems.
\end{itemize}

\textbf{Strict Constraints from the Assignment Brief:}

\noindent
\begin{itemize}
    \item \textbf{Context:} The search must align with the ``Flora \& Fauna'' category as defined in Kandris, D. et al. (2020), ``Applications of Wireless Sensor Networks: An Up-to-Date Survey''.
    \item \textbf{Date Range:} Prioritize papers from 2021--2025. Do not suggest anything older than 2015.
    \item \textbf{Paper Quality:} Focus on Journal Articles from reputable publishers (IEEE, Elsevier/ScienceDirect, MDPI Sensors, ACM). I have access to University Google Scholar and IEEE library among others, so I can access wide range of papers.
    \item \textbf{Technical Depth:} Each suggested case must be based on a paper that describes a concrete implementation or architecture. The paper must contain enough detail to analyze:
    \begin{itemize}
        \item \textbf{Architecture \& Technologies:} (e.g., LoRaWAN, NB-IoT, ZigBee, 5G, or acoustic underwater protocols). Maybe even Smart Dust.
        \item \textbf{QoS \& Performance:} The paper should mention data like battery life (energy consumption), scalability, or network management.
    \end{itemize}
\end{itemize}

\noindent
\textbf{Output Format:} For each suggestion, provide:

\begin{itemize}
    \item Title of the Research Paper \& Authors.
    \item Publication Year \& Journal Name.
    \item A Brief ``Hook'': Why is this case unique/rare?
    \item Data Availability Check: Briefly confirm if the paper includes a description of the system architecture or technical parameters (QoS, lifetime, etc.) needed for my analysis.
\end{itemize}

\chapter{Deep Research löydökset}
\label{app:deep-research-results}

Viimeinen sarake, \emph{Grade}, on tämän raportin kirjoittajan antama paino sille, kuinka hyvin julkaisu soveltuu tehtävänantoon ja mieltymyksiin. Arvo on määritelty ensimmäisellä lukukerralla. Toisen, tarkemman lukukerran myötä tippuivat loput, kuten 9 ja 9B. Suluissa oleva teksti on mahdollinen syy ensi käden hylkäykselle, kuten jos teos on selkeästi IoT- tai koneoppimisaiheinen.

\begin{longtable}{|p{0.8cm}|p{6cm}|p{1.5cm}|p{1.8cm}|p{2cm}|}
\hline
\textbf{Nro} & \textbf{Alkuperäinen otsikko} & \textbf{Vuosi} & \textbf{Linkki} & \textbf{Grade \%} \\
\hline
\endfirsthead

\multicolumn{5}{c}%
{{\bfseries -- jatkuu edelliseltä sivulta}} \\
\hline
\textbf{Nro} & \textbf{Alkuperäinen otsikko} & \textbf{Vuosi} & \textbf{Linkki} & \textbf{Grade \%} \\
\hline
\endhead

\hline \multicolumn{5}{|r|}{{Jatkuu seuraavalla sivulla...}} \\ \hline
\endfoot

\hline
\endlastfoot
1 & Internet of Plants: Machine Learning System for Bioimpedance-Based Plant Monitoring & 2025 & \href{https://www.mdpi.com/1424-8220/25/24/7549}{MPDI} & 50 (ML) \% \\
\hline
2 & I-TREES: A Context-Aware Framework for Energy-Efficient Tree Health Monitoring in Forest Internet of Trees (IoTr) Networks & 2025 & \href{https://ojs.aaai.org/index.php/AAAI-SS/article/view/36043}{AAAI} & 60 \% (IoT) \\
\hline
3 & Hearing nature's heartbeat: towards large-scale real-time forest monitoring network in Italy & 2025 & \href{https://iforest.sisef.org/contents/?id=ifor4830-018}{iForest} & 30 \% (IoT) \\
\hline
4 & Toward a Unified TreeTalker Data Curation Process & 2022 & \href{https://www.mdpi.com/1999-4907/13/6/855}{MDPI} & 60 \% (IoT) \\
\hline
5 & Fungal circuitry: mycelium as a living sensor for smart structures & 2024 & \href{https://www.researchgate.net/publication/380473735_Fungal_circuitry_mycelium_as_a_living_sensor_for_smart_structures}{Linkki} & 0 \% (Off topic) \\
\hline
6 & Enhancing Vanilla Planifolia Generative Phase via IoT-Based Microclimate Control & 2025 & \href{https://ejournal.unipas.ac.id/index.php/Agro/article/view/2171}{Unipas} & 0 \% (IoT) \\
\hline
7 & Chatting with Plants (Orchids) in Automated Smart Farming using IoT, Fuzzy Logic and Chatbot & 2019 & \href{https://www.astesj.com/v04/i05/p22/}{ASTEJ} & 0 \% (IoT) \\
\hline
8 & WaggleNet: A LoRa and MQTT-Based Monitoring System for Internal and External Beehive Conditions & 2025 & \href{https://www.arxiv.org/pdf/2512.07408}{ArXiv} & 90 \% \\
\hline
9 & Intelligent IoT-Aided Early Sound Detection of Red Palm Weevils & 2021 & \href{https://www.researchgate.net/publication/354148479_Intelligent_IoT-Aided_Early_Sound_Detection_of_Red_Palm_Weevils}{ResearchG} & 80 \% \\
\hline
9B & TreeVibes: Modern Tools for Global Monitoring of Trees for Borers & 2021 & \href{https://www.mdpi.com/2624-6511/4/1/17}{MDPI} & 90 \% \\
\hline
10 & Early Detection of Red Palm Weevil Infestations using Deep Learning Classification of Acoustic Signals & 2023 & \href{https://arxiv.org/abs/2308.15829}{ArXiv} & 0 \% (ML) \\
\hline
11 & MosquIoT: A System Based on IoT and Machine Learning for the Monitoring of Aedes aegypti & 2024 & \href{https://arxiv.org/abs/2401.16258}{ArXiv} & 0 \% (ML) \\
\hline
12 & Embracing firefly flash pattern variability with data-driven species classification & 2023 & \href{https://www.biorxiv.org/content/10.1101/2023.03.08.531653v3.full-text}{BiorXiv} & 0 \% (ML) \\
\hline
13 & Timely poacher detection and localization using sentinel animal movement & 2021 & \href{https://www.nature.com/articles/s41598-021-83800-1}{Nature} & 0 \% (GPS, ML) \\
\hline
14 & Monitoring Endangered and Rare Wildlife in the Field: A Foundation Deep Learning Model (KI-CLIP) & 2023 & \href{https://www.mdpi.com/2076-2615/13/20/3168}{MPDI} & 0 \% (ML) \\
\hline
15 & AviEar: An IoT-based Low Power Solution for Acoustic Monitoring of Avian Species & 2024 & \href{https://ieeexplore.ieee.org/abstract/document/10742275}{IEEE} & 0 \% (IoT) \\
\hline
16 & TRITON: Open Telemetry and Location Estimation for Marine Monitoring Based on IoT and LoRa & 2024 & \href{https://ieeexplore.ieee.org/abstract/document/10807074}{IEEE} & 10 \% (IoT) \\
\hline
17 & Energy-optimized monitoring system for underground cave environments based on long preamble LoRa & 2025 & \href{https://ieeexplore.ieee.org/document/10870271}{IEEE} & 80 \% \\
\hline
18 & An Integrated UIoT-based Coral Reef Monitoring and Protection Platform Based on AI \& Digital Twin & 2023 & \href{https://www.researchgate.net/publication/368247076_AN_INTEGRATED_UIOT-BASED_CORAL_REEF_MONITORING_AND_PROTECTION_PLATFORM_BASED_ON_AI_DIGITAL_TWIN}{Linkki} & 0 \% (PowerPoint) \\
\hline
19 & Smart Aquaculture Analytics: Enhancing Shrimp Farming in Bangladesh through Real-Time IoT Monitoring & 2024 & \href{https://www.researchgate.net/publication/383680323_Smart_Aquaculture_Analytics_Enhancing_Shrimp_Farming_in_Bangladesh_through_Real-Time_IoT_Monitoring_and_Predictive_Machine_Learning_Analysis}{Linkki} & 0 \% (IoT, ML) \\
\hline
20 & An Internet of Things (IoT) Application on Volcano Monitoring & 2019 & \href{https://www.mdpi.com/1424-8220/19/21/4651}{MDPI} & 90 \% \\
\hline
\end{longtable}

\chapter{Google Scholar haut}
\label{app:google-scholar-results}

Perinteisen Google Scholar -haun käytetyt hakusanat. Kaikissa haiussa oli rajauksena vuosi 2022 tai sitä tuoreemmat julkaisut.

\begin{itemize}
    \item \textbf{A:} \texttt{"WSN"\ AND\ "precision agriculture"}
    \item \textbf{B:} \texttt{"WSN"\ AND\ "endangered"}
    \item \textbf{C:} \texttt{"WSN"\ AND\ "Wildlife"}
    \item \textbf{D:} \texttt{"WSN"\ AND\ "fungal"\ AND\ "sensor"}
\end{itemize}

\begin{longtable}{|p{0.8cm}|p{1cm}|p{7cm}|p{1.8cm}|}
\hline
\textbf{Nro} & \textbf{Haku} & \textbf{Otsikko} & \textbf{Linkki} \\
\hline
\endfirsthead

\multicolumn{5}{c}%
{{\bfseries -- jatkuu edelliseltä sivulta}} \\
\hline
\textbf{Nro} & \textbf{Haku} & \textbf{Otsikko} & \textbf{Linkki} \\
\hline
\endhead

\hline \multicolumn{5}{|r|}{{Jatkuu seuraavalla sivulla...}} \\ \hline
\endfoot

\hline
\endlastfoot
21 & A & Low-Power Intelligent Wireless Sensor Network for Precision Agriculture Oriented Agricultural Greenhouse Management System & \href{https://ieeexplore.ieee.org/document/10891530}{IEEE} \\
\hline
22 & A & Wireless Sensor Networks for Precision Agriculture: A Review of NPK Sensor Implementations & \href{https://www.mdpi.com/1424-8220/24/1/51}{MDPI} \\
\hline
23 & B & WSN Protocols and Security Challenges for Environmental Monitoring Applications: A Survey & \href{https://onlinelibrary.wiley.com/doi/abs/10.1155/2022/1628537}{Wiley} \\
\hline
24 & B & Locating Nesting Sites for Critically Endangered Galápagos Pink Land Iguanas (Conolophus marthae) & \href{https://www.mdpi.com/2076-2615/14/12/1835}{MDPI} \\
\hline
25 & C & Unmanned Aerial Surveillance and Tracking System in Forest Areas for Poachers and Wildlife & \href{https://ieeexplore.ieee.org/abstract/document/10788704}{IEEE} \\
\hline
26 & D & Location-Aware IoT-Enabled Wireless Sensor Networks for Landslide Early Warning & \href{https://www.mdpi.com/2079-9292/11/23/3971}{MDPI} \\
\hline
\end{longtable}

\bibliography{KokkolaRaportti}

\end{document}
