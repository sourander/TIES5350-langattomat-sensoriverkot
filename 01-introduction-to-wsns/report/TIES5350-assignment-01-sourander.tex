\documentclass[finnish,utf8,nonumbib,palatino]{KokkolaRaportti}

\usepackage[bookmarksopen,bookmarksnumbered]{hyperref}

% ------------ Omat alkaa ------------
\usepackage{fontspec}
\setmainfont{TeX Gyre Pagella}

% For minted (pygment code syntax highlighting)
\usepackage[outputdir=aux]{minted}

% Minted background color
\definecolor{LightGray}{gray}{0.95}

% Potentially set some defaults, like
\setminted{
  % These two are inluded in my tex-code-minted-block snippet. tex
  % frame=lines,
  % linenos,               % Show line numbers
  bgcolor=LightGray,
  breaklines,            % Wrap long lines
  autogobble,            % Automatically dedent leadings whitespace
  fontsize=\footnotesize % Font size to footnotesize. Alternative e.g. \small.
}

% Colors for TODO-list
\usepackage{xcolor}
\newcommand{\todo}{\textcolor{orange}{TODO }}
\newcommand{\done}{\textcolor{green}{DONE }}
\newcommand{\miss}{\textcolor{red}{MISS }}

% ------------ Omat loppuu ------------

\title{Mallipohja raportin kirjoittamiseen Latex:lla}
\setauthor{Jani}{Sourander}

\begin{document}

\mainmatter

% --------------------------------------------------  %
%                                                     %
%                  Johdanto                           %
%                                                     %
% --------------------------------------------------- %
\chapter{Johdanto}

Valittu vaihtoehto, A, on teemaltaan \emph{Minun IoT-systeemini}.

\subsection*{Declaration of Using Generative Artificial Intelligence}
No generative AI was used in the preparation of this assignment.


% --------------------------------------------------  %
%                                                     %
%                  Arkkitehtuuri                      %
%                                                     %
% --------------------------------------------------- %
\chapter{Arkkitehtuuri}

Mistä komponenteista ja palveluista järjestelmä koostuu/koostuisi?

\bibliography{KokkolaRaportti}

\end{document}
