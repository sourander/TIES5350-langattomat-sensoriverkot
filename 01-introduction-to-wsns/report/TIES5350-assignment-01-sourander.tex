\documentclass[finnish,utf8,nonumbib,palatino]{KokkolaRaportti}

\usepackage[bookmarksopen,bookmarksnumbered]{hyperref}

% ------------ Omat alkaa ------------
\usepackage{fontspec}
\setmainfont{TeX Gyre Pagella}

% For minted (pygment code syntax highlighting)
\usepackage[outputdir=aux]{minted}

% For long tables that span multiple pages
\usepackage{longtable}

% Minted background color
\definecolor{LightGray}{gray}{0.95}

% Potentially set some defaults, like
\setminted{
  % These two are inluded in my tex-code-minted-block snippet. tex
  % frame=lines,
  % linenos,               % Show line numbers
  bgcolor=LightGray,
  breaklines,            % Wrap long lines
  autogobble,            % Automatically dedent leadings whitespace
  fontsize=\footnotesize % Font size to footnotesize. Alternative e.g. \small.
}

% Colors for TODO-list
\usepackage{xcolor}
\newcommand{\todo}{\textcolor{orange}{TODO }}
\newcommand{\done}{\textcolor{green}{DONE }}
\newcommand{\miss}{\textcolor{red}{MISS }}

% ------------ Omat loppuu ------------

\title{Mallipohja raportin kirjoittamiseen Latex:lla}
\setauthor{Jani}{Sourander}

\begin{document}

\mainmatter

% --------------------------------------------------  %
%                                                     %
%                  Johdanto                           %
%                                                     %
% --------------------------------------------------- %
\chapter{Johdanto}


Tämä raportti on osa Langattomat sensoriverkot -kurssia. Tehtänantona on valita yksi Kandris et al. (2020) esittämistä luokituksista. Kuusi sovellusalueesta tässä työssä keskitytään erityisesti ympäristö (engl. Environmental) -alueeseen, jossa sensoriverkkoja hyödynnetään maaperän, veden tai ilman seurannassa.

Raportissa tarkastellaan kahta tieteellistä artikkelia kurssin luentomateriaalien ja -videoiden rajaamien aiheiden kautta.. Ensimmäinen, \emph{An Internet of Things (IoT) Application on Volcano Monitoring} (Awadallah et al., 2019) \cite{volcano}, esittelee konkreettisen ympäristön monitorointisovelluksen, jossa LoRa-verkkoteknologiaa ja virtapihejä sensorisolmuja hyödynnetään tulivuoren maaperän lämpötilan seurannassa. Raportti tutustuu artikkeli kautta sen arkkitehtuuriin, laitteistoon, viestintäprotokolliin, energiatehokkuuden ja turvallisuuteen. Sovellus edustaa ympäristöluokan (Environmental) sensoriverkkoja ja auttaa ennustamaan tulivuorenpurkauksia.

Toinen artikkeli, \emph{WSN Protocols and Security Challenges for Environmental Monitoring Applications: A Survey} (Adu-Manu et al., 2022) \cite{security}, toimii tausta-aineistona ja tarjoaa laajan katsauksen ympäristön monitorointiin liittyviin protokolliin ja tietoturvahaasteisiin. Tämä kirjallisuuskatsaus täydentää sovellusartikkelin tarkastelua tuomalla esiin turvallisuuteen liittyviä näkökulmia.

\subsection*{Generatiivisen tekoälyn käyttö}

Tekoälyä on käytetty tiedonhaun apuna. Suoritin sekä perinteistä Google Scholar että Gemini Deep Research tiedonhakua. Jälkimmäisessä käytetty kehote ja taulukkomuotoon parsimani tulokset löytyvät liitteistä (ks. Liite \ref{app:deep-research} ja Liite \ref{app:deep-research-results}). Tekoälyä on käytetty myös LaTeX-muotoilun apuna taulukoissa.


% --------------------------------------------------  %
%                                                     %
%                 Volcano                             %
%                                                     %
% --------------------------------------------------- %
\chapter{Tulivuorenpurkauksen ennustaminen}


% --------------------------------------------------  %
%                Artikkelin valinta                   %
% --------------------------------------------------- %
\section{Artikkeleiden haku ja valinta}

Raporttia kirjoittaessa oli tärkeää löytää artikkeli (tai useampi), jossa kuvataan sovellusta eri arkkitehtuurin kerrosten näkökulmista. Tätä valintaa vaikeuttaa se, että kurssin aihe, langattomat sensoriverkot (\emph{engl. Wireless Sensor Networks, WSN}) on hyvin lähellä risteäviä aiheita kuten esineiden internetiä (\emph{engl. Internet of Things, IoT}), joten sopivaa työtä ei välttämättä tunnista otsikosta tai tiivistelmästä. Toinen vaikeuttava tekijä on, että kurssin moduulin katsantakulma on laaja, kattaen vaiheet mittauskerroksesta applikaatiokerrokseen asti. Tämä asettaa laajuusvaatimuksia löydetylle julkaisulle. Näiden haasteiden johdosta tiedonhaun prosessiksi valikoitui tapa, jossa ladataan kymmeniä PDF-julkaisua koneelle, ja nämä käydään silmäillen läpi, karsien ne pois, jotka eivät läpäise vaatimuksia. Haku suoritettiin sekä Google Deep Research että Google Scholar palveluita käyttäen ja hakuprosessi esitellään raportin liitteissä.

Kuten liitteiden kehotteista on luettavissa, alkuperäinen aiherajaus oli \emph{Flora \& Fauna}. Julkaisuja karsiessa valtaosa tekoälyn ehdottamista julkaisuista karsiutui pois, koska niiden pääpaino oli selkeästi IoT:n tai koneoppimisen puolella. Pisimpään harkinnanalaisena pysyi TreeVibes-sensoriin liittyvät kaksi  julkaisua. Ne tunnetaan numeroilla 9 ja 9B liitteissä (ks.Liite \ref{app:deep-research-results}). Deep Resarchin tuottama ehdotus ei sisältänyt alkuperäistä TreeVibes-julkaisua, mutta julkaisu tunnusnumeroltaan 9 käytti kyseistä sensoria ja viittasi tähän työhön lähteissään. Artikkeleiden tarkempi silmäily valitettavasti todisti, että TreeVibes on pikemminkin IoT- kuin WSN-toteutus, joten julkaisut karsiutuivat pois. Tähän mennessä kaikki ladatut PDF-tiedostot olivat joutuneet hylkäyspinoon, paitsi väärää sovellusaluetta, eli ympäristöä, -- tarkemmin tulivuoria -- käsittelevä työ. Lopulta ei ollut muuta vaihtoehtoa kuin tehdä pivot: sovellusalueeksi valikoitui \emph{Environmental}.

Näin sovellukseksi oli valikoitunut kirjoittajien Awadallah, Moure ja Torres-González julkaisu \emph{An Internet of Things (IoT) Application on Volcano Monitoring}. Julkaisu on Espanjan kansallisen maantieteellisen instituutin (IGN, esp. \emph{Instituto Geográfico Nacional}) Karariansaarten geofysiikan keskukselta (esp. \emph{Centro Geofísico de Canarias}). Myös tässä julkaisussa mainitaan IoT useita kertoja -- alkaen otsikosta. Luennoilla Ismo Hakala käsitteli tätä aihetta: termit Wireless Sensor Networks (WSN), Internet of Things (IoT) sekä Cyber-Physical Systems (CPS) ovat läheisesti toisiinsa liittyviä käsitteitä. Sopivalla Google Scholar -haulla (\texttt{"WSN"\ AND\ "IoT-enabled"}) löytyy vuoden 2020 jälkeen yli $13\,000$ tulosta. Kenties IoT nähdään tällöin IP-protokollaperheen tarjoamana toiminnallisuutena, jonka alemmalla kerroksella voi olla tai voi olla olematta kurssin määritelmän mukaisia WSN-laitteita. Käsitellyssä julkaisussa IoT-tasoa edustaa Raspberry Pi -gatewayltä alkava Internet-yhteys. Alemman mittaustason laitteet ovat kurssin määritelmän mukaisia WSN-mittalaitteita ja hyödyntävät kommunikaatioon LoRa-tekniikkaa. Näihin teknisiin seikkoihin tutustutaan myöhemmissä luvuissa.

Valitun sovelluksen lisäksi raportin aineistoksi valikoitui toinen julkaisu, joka on kirjallisuuskatsaus (engl. review article), joka käsittelee WSN-protokollien turvallisuutta. Kyseinen Google Scholarilla löytynyt lähde, joka tunnetaan liitteissä tunnusnumerolla 26(ks. Liite \ref{app:google-scholar-results}), on otsikoltaan \emph{WSN Protocols and Security Challenges for Environmental Monitoring Applications: A Survey} \cite{security}. Tämä katsaus on mukana vertailukohtana ja tarjoamassa laajempaa pohjaa turvallisuusaiheen käsittelylle.

% --------------------------------------------------  %
%                Yleiskuvaus                          %
% --------------------------------------------------- %
\section{Sovellusartikkeleiden yleiskatsaus}

\todo Kuvaa lyhyesti sovelluksen taustat: minkä ongelman se ratkaisee?

\todo Artikkelissa A keskitytään... 


% --------------------------------------------------  %
%          Toiminnallisuus ja teknologiat             %
% --------------------------------------------------- %
\section{Toiminnallisuus ja teknologiat}

\todo Tässä on tärkeää käyttää luennoilla opittua luokittelua (arkkitehtuuri, protokollat, teknologiat).

\todo Yritä selvittää/päätellä QoS, elinaika, laajennettavuus, ylläpidettävyys, ohjelmoitavuus, tietoturvallisuus, energiankulutus, itseorganisoituvuus, verkonhallinta jne

\todo Voit käyttää taulukkoa, jos haluat havainnollistaa eroja (esim. Artikkeli A käyttää LoRa-teknologiaa, Artikkeli B ZigBeetä).

\todo Käy läpi vaaditut tekniset termit: QoS, elinaika (energiankulutus), ylläpidettävyys ja skaalautuvuus kummankin paperin osalta.

% --------------------------------------------------  %
%               Analyysi ja vertailu                  %
% --------------------------------------------------- %
\section{Analyysi ja vertailu}

\todo Vertaile näkökulmasta: QoS, elinaika, laajennettavuus, ylläpidettävyys, ohjelmoitavuus, tietoturvallisuus, energiankulutus, itseorganisoituvuus, verkonhallinta jne

\todo Arvioi, kuinka hyvin valitut toteutukset vastaavat niille asetettuja tavoitteita. Koska sinulla on kaksi artikkelia, voit analysoida, miksi toisessa on valittu erilainen arkkitehtuuri kuin toisessa.

% --------------------------------------------------  %
%                                                     %
%                  Yhteenveto                         %
%                                                     %
% --------------------------------------------------- %
\chapter{Yhteenveto}

\todo Tähän sisältöä.

\appendix

\chapter{Deep Research kehote}
\label{app:deep-research}

Alla on kehote, jota käytin löytääkseni mielenkiintoisia WSN-sovelluskohteita valitussa kontekstissa. Tekoälynä Google Gemini Pro, Deep Reseach -tila.

\vspace{1.0\baselineskip} % ----

\noindent
\textbf{Role:} You are an expert academic researcher specializing in Wireless Sensor Networks (WSN) and the Internet of Things (IoT).

\vspace{1.0\baselineskip} % ----

\noindent
\textbf{Task:} Identify 10--20 unique, rare, or surprising ``Flora \& Fauna'' WSN application cases. I am looking for ``out-of-the-box'' topics that go beyond common greenhouse monitoring or livestock tracking.

\vspace{1.0\baselineskip} % ----

\noindent
\textbf{Inspiration for Uniqueness:}

\noindent
\begin{itemize}
    \item \textbf{Fauna:} Monitoring the synchronized flashing of fireflies, tracking highly endangered or elusive species (e.g., pangolins, snow leopards, or deep-sea vents), or bio-logging with miniature sensors on insects/birds.
    \item \textbf{Flora:} Monitoring ``plant intelligence'' or communication (e.g., fungal networks/mycelium), monitoring rare orchid habitats, or tracking the ``heartbeat'' (sap flow) of ancient trees.
    \item \textbf{Extreme (Flora or Fauna) Environments:} Underwater sensor networks for coral reef health or sensors in high-altitude volcanic ecosystems.
\end{itemize}

\textbf{Strict Constraints from the Assignment Brief:}

\noindent
\begin{itemize}
    \item \textbf{Context:} The search must align with the ``Flora \& Fauna'' category as defined in Kandris, D. et al. (2020), ``Applications of Wireless Sensor Networks: An Up-to-Date Survey''.
    \item \textbf{Date Range:} Prioritize papers from 2021--2025. Do not suggest anything older than 2015.
    \item \textbf{Paper Quality:} Focus on Journal Articles from reputable publishers (IEEE, Elsevier/ScienceDirect, MDPI Sensors, ACM). I have access to University Google Scholar and IEEE library among others, so I can access wide range of papers.
    \item \textbf{Technical Depth:} Each suggested case must be based on a paper that describes a concrete implementation or architecture. The paper must contain enough detail to analyze:
    \begin{itemize}
        \item \textbf{Architecture \& Technologies:} (e.g., LoRaWAN, NB-IoT, ZigBee, 5G, or acoustic underwater protocols). Maybe even Smart Dust.
        \item \textbf{QoS \& Performance:} The paper should mention data like battery life (energy consumption), scalability, or network management.
    \end{itemize}
\end{itemize}

\noindent
\textbf{Output Format:} For each suggestion, provide:

\begin{itemize}
    \item Title of the Research Paper \& Authors.
    \item Publication Year \& Journal Name.
    \item A Brief ``Hook'': Why is this case unique/rare?
    \item Data Availability Check: Briefly confirm if the paper includes a description of the system architecture or technical parameters (QoS, lifetime, etc.) needed for my analysis.
\end{itemize}

\chapter{Deep Research löydökset}
\label{app:deep-research-results}

Viimeinen sarake, \emph{Grade}, on tämän raportin luojan antama paino sille, kuinka hyvin julkaisu soveltuu tehtävänantoon ja mieltymyksiin. Suluissa oleva teksti on mahdollinen syy hylkäykselle, kuten jos teos on selkeästi IoT- tai koneoppimisaiheinen.

\begin{longtable}{|p{0.8cm}|p{6cm}|p{1.5cm}|p{1.8cm}|p{2cm}|}
\hline
\textbf{Nro} & \textbf{Alkuperäinen otsikko} & \textbf{Vuosi} & \textbf{Linkki} & \textbf{Grade \%} \\
\hline
\endfirsthead

\multicolumn{5}{c}%
{{\bfseries -- jatkuu edelliseltä sivulta}} \\
\hline
\textbf{Nro} & \textbf{Alkuperäinen otsikko} & \textbf{Vuosi} & \textbf{Linkki} & \textbf{Grade \%} \\
\hline
\endhead

\hline \multicolumn{5}{|r|}{{Jatkuu seuraavalla sivulla...}} \\ \hline
\endfoot

\hline
\endlastfoot
1 & Internet of Plants: Machine Learning System for Bioimpedance-Based Plant Monitoring & 2025 & \href{https://www.mdpi.com/1424-8220/25/24/7549}{MPDI} & 50 \% \\
\hline
2 & I-TREES: A Context-Aware Framework for Energy-Efficient Tree Health Monitoring in Forest Internet of Trees (IoTr) Networks & 2025 & \href{https://ojs.aaai.org/index.php/AAAI-SS/article/view/36043}{AAAI} & 60 \% (IoT) \\
\hline
3 & Hearing nature's heartbeat: towards large-scale real-time forest monitoring network in Italy & 2025 & \href{https://iforest.sisef.org/contents/?id=ifor4830-018}{iForest} & 30 \% (IoT) \\
\hline
4 & Toward a Unified TreeTalker Data Curation Process & 2022 & \href{https://www.mdpi.com/1999-4907/13/6/855}{MDPI} & 60 \% (IoT) \\
\hline
5 & Fungal circuitry: mycelium as a living sensor for smart structures & 2024 & \href{https://www.researchgate.net/publication/380473735_Fungal_circuitry_mycelium_as_a_living_sensor_for_smart_structures}{Linkki} & 0 \% (Off topic) \\
\hline
6 & Enhancing Vanilla Planifolia Generative Phase via IoT-Based Microclimate Control & 2025 & \href{https://ejournal.unipas.ac.id/index.php/Agro/article/view/2171}{Unipas} & 0 \% (IoT) \\
\hline
7 & Chatting with Plants (Orchids) in Automated Smart Farming using IoT, Fuzzy Logic and Chatbot & 2019 & \href{https://www.astesj.com/v04/i05/p22/}{ASTEJ} & 0 \% (IoT) \\
\hline
8 & WaggleNet: A LoRa and MQTT-Based Monitoring System for Internal and External Beehive Conditions & 2025 & \href{https://www.arxiv.org/pdf/2512.07408}{ArXiv} & 90 \% \\
\hline
9 & Intelligent IoT-Aided Early Sound Detection of Red Palm Weevils & 2021 & \href{https://www.researchgate.net/publication/354148479_Intelligent_IoT-Aided_Early_Sound_Detection_of_Red_Palm_Weevils}{ResearchG} & 80 \% \\
\hline
9B & TreeVibes: Modern Tools for Global Monitoring of Trees for Borers & 2021 & \href{https://www.mdpi.com/2624-6511/4/1/17}{MDPI} & 90 \% \\
\hline
10 & Early Detection of Red Palm Weevil Infestations using Deep Learning Classification of Acoustic Signals & 2023 & \href{https://arxiv.org/abs/2308.15829}{ArXiv} & 0 \% (ML) \\
\hline
11 & MosquIoT: A System Based on IoT and Machine Learning for the Monitoring of Aedes aegypti & 2024 & \href{https://arxiv.org/abs/2401.16258}{ArXiv} & 0 \% (ML) \\
\hline
12 & Embracing firefly flash pattern variability with data-driven species classification & 2023 & \href{https://www.biorxiv.org/content/10.1101/2023.03.08.531653v3.full-text}{BiorXiv} & 0 \% (ML) \\
\hline
13 & Timely poacher detection and localization using sentinel animal movement & 2021 & \href{https://www.nature.com/articles/s41598-021-83800-1}{Nature} & 0 \% (GPS, ML) \\
\hline
14 & Monitoring Endangered and Rare Wildlife in the Field: A Foundation Deep Learning Model (KI-CLIP) & 2023 & \href{https://www.mdpi.com/2076-2615/13/20/3168}{MPDI} & 0 \% (ML) \\
\hline
15 & AviEar: An IoT-based Low Power Solution for Acoustic Monitoring of Avian Species & 2024 & \href{https://ieeexplore.ieee.org/abstract/document/10742275}{IEEE} & 0 \% (IoT) \\
\hline
16 & TRITON: Open Telemetry and Location Estimation for Marine Monitoring Based on IoT and LoRa & 2024 & \href{https://ieeexplore.ieee.org/abstract/document/10807074}{IEEE} & 10 \% (IoT) \\
\hline
17 & Energy-optimized monitoring system for underground cave environments based on long preamble LoRa & 2025 & \href{https://ieeexplore.ieee.org/document/10870271}{IEEE} & 80 \% \\
\hline
18 & An Integrated UIoT-based Coral Reef Monitoring and Protection Platform Based on AI \& Digital Twin & 2023 & \href{https://www.researchgate.net/publication/368247076_AN_INTEGRATED_UIOT-BASED_CORAL_REEF_MONITORING_AND_PROTECTION_PLATFORM_BASED_ON_AI_DIGITAL_TWIN}{Linkki} & 0 \% (PowerPoint) \\
\hline
19 & Smart Aquaculture Analytics: Enhancing Shrimp Farming in Bangladesh through Real-Time IoT Monitoring & 2024 & \href{https://www.researchgate.net/publication/383680323_Smart_Aquaculture_Analytics_Enhancing_Shrimp_Farming_in_Bangladesh_through_Real-Time_IoT_Monitoring_and_Predictive_Machine_Learning_Analysis}{Linkki} & 0 \% (IoT, ML) \\
\hline
20 & An Internet of Things (IoT) Application on Volcano Monitoring & 2019 & \href{https://www.mdpi.com/1424-8220/19/21/4651}{MDPI} & 90 \% \\
\hline
\end{longtable}

\chapter{Google Scholar haut}
\label{app:google-scholar-results}

Perinteisen Google Scholar -haun käytetyt hakusanat. Kaikissa haiussa oli rajauksena vuosi 2022 tai sitä tuoreemmat julkaisut.

\begin{itemize}
    \item \textbf{A:} \texttt{"WSN"\ AND\ "precision agriculture"}
    \item \textbf{B:} \texttt{"WSN"\ AND\ "endangered"}
    \item \textbf{C:} \texttt{"WSN"\ AND\ "Wildlife"}
    \item \textbf{D:} \texttt{"WSN"\ AND\ "fungal"\ AND\ "sensor"}
\end{itemize}

\begin{longtable}{|p{0.8cm}|p{1cm}|p{7cm}|p{1.8cm}|}
\hline
\textbf{Nro} & \textbf{Haku} & \textbf{Otsikko} & \textbf{Linkki} \\
\hline
\endfirsthead

\multicolumn{5}{c}%
{{\bfseries -- jatkuu edelliseltä sivulta}} \\
\hline
\textbf{Nro} & \textbf{Haku} & \textbf{Otsikko} & \textbf{Linkki} \\
\hline
\endhead

\hline \multicolumn{5}{|r|}{{Jatkuu seuraavalla sivulla...}} \\ \hline
\endfoot

\hline
\endlastfoot
21 & A & Low-Power Intelligent Wireless Sensor Network for Precision Agriculture Oriented Agricultural Greenhouse Management System & \href{https://ieeexplore.ieee.org/document/10891530}{IEEE} \\
\hline
22 & A & Wireless Sensor Networks for Precision Agriculture: A Review of NPK Sensor Implementations & \href{https://www.mdpi.com/1424-8220/24/1/51}{MDPI} \\
\hline
23 & B & WSN Protocols and Security Challenges for Environmental Monitoring Applications: A Survey & \href{https://onlinelibrary.wiley.com/doi/abs/10.1155/2022/1628537}{Wiley} \\
\hline
24 & B & Locating Nesting Sites for Critically Endangered Galápagos Pink Land Iguanas (Conolophus marthae) & \href{https://www.mdpi.com/2076-2615/14/12/1835}{MDPI} \\
\hline
25 & C & Unmanned Aerial Surveillance and Tracking System in Forest Areas for Poachers and Wildlife & \href{https://ieeexplore.ieee.org/abstract/document/10788704}{IEEE} \\
\hline
26 & D & Location-Aware IoT-Enabled Wireless Sensor Networks for Landslide Early Warning & \href{https://www.mdpi.com/2079-9292/11/23/3971}{MDPI} \\
\hline
\end{longtable}

\bibliography{KokkolaRaportti}

\end{document}
